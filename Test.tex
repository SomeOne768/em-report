\section{Tests}

Testing is a critical phase in software development, aimed at ensuring the quality, reliability, and robustness of the code. It helps in identifying errors, bugs, and potential issues before the software is deployed to production. There are various types of tests, each serving a specific purpose within the software development lifecycle. Unit tests and integration tests are among the most common approaches, each playing a vital role in maintaining software quality and are the 2 currently implemented in the application.\\
Testing also allow us to maintain the application by ensuring that there is no regression. Indeed, sometimes by adding a functionnality or modifying the code we may break the application or adding bug.


\subsection{Unit Tests}
Unit tests focus on verifying the functionality of individual components or pure functions. These tests are designed to ensure that each piece of code behaves as expected under various conditions. Unit tests are typically written and run by developers during the development process to catch errors early and maintain code quality. By testing the smallest units of code independently, unit tests provide a strong foundation for reliable software.

\subsection{Integration Tests}
Integration tests evaluate the interaction between multiple components or systems to ensure they work together as intended. These tests are crucial for identifying issues that may arise when different parts of the application are combined. Integration tests can involve testing the interaction between modules, services, or external APIs. By simulating real-world scenarios, integration tests help ensure that the entire system functions cohesively and meets the desired requirements.
