\section{Glossary}

\textbf{API:} Application Programming Interface. A set of rules and definitions that allows different software applications to communicate with each other.\\

\textbf{AWS:} Amazon Web Services. A comprehensive cloud platform offering various services such as computing power, storage, and networking.\\

\textbf{Back-end:} The server-side part of an application that handles business logic, database interactions, and server communication.\\

\textbf{Branch:} In version control systems, a branch is an independent line of development that allows multiple developers to work on different features simultaneously.\\

\textbf{CI/CD:} Continuous Integration/Continuous Deployment. Practices in software engineering that ensure automated testing, integration, and deployment of applications.\\

\textbf{Command execution:} The process of running a specific command to directly order the OS in an operating system or application environment.\\

\textbf{CPU:} Central Processing Unit. The primary component of a computer that performs most of the processing inside a computer.\\

\textbf{CRUD:} Acronym for Create, Read, Update, Delete. It represents the four basic operations of persistent storage in database management systems and software applications.

\textbf{CSS:} Cascading Style Sheets. A style sheet language used to describe the presentation and design of a document written in HTML or XML.\\

\textbf{Database:} An organized collection of data, typically stored electronically in a structured format.\\

\textbf{DBMS:} Database Management System. Software that interacts with databases to manage and query data.\\

\textbf{Design pattern:} A general, reusable solution to a commonly occurring problem within a given context in software design.\\

\textbf{Docker:} An open-source platform that automates the deployment of applications inside lightweight, portable containers.\\

\textbf{DTO:} Data Transfer Object. A design pattern used to transfer data between software application subsystems.\\

\textbf{Elastic Search:} A distributed search and analytics engine commonly used for indexing large volumes of data and for text search.\\

\textbf{Environments:} Different setups or configurations where software applications are developed, tested, and deployed (e.g., development, local, production).\\

\textbf{Filters:} In Spring Security, filters are mechanisms used to process requests and responses within the web application security framework. They are used to perform tasks such as authentication, authorization, logging, and data transformation based on specific criteria. Filters are part of the request processing chain and can modify or inspect requests and responses before they reach the application or after they leave it.\\

\textbf{Frameworks:} Predefined structures or platforms for developing software applications that provide reusable code and tools.\\

\textbf{Front-end:} The client-side part of an application that deals with the user interface and user experience (UI/UX).\\

\textbf{Heroku:} A cloud platform that enables developers to build, run, and scale applications quickly without managing infrastructure.\\

\textbf{HTML:} Hyper Text Markup Language. The standard language used to create web pages and web applications.\\

\textbf{IDE:} Integrated Development Environment. Software providing comprehensive tools (e.g., code editor, debugger) for developers to write and test code.\\

\textbf{JSON:} JavaScript Object Notation. A lightweight data-interchange format used for storing and exchanging structured data.\\

\textbf{Libraries:} Sets of code that developers can use to optimize and simplify tasks in software development.\\

\textbf{Minio:} An open-source, high-performance object storage system compatible with Amazon S3.\\

\textbf{MVC:} Model-View-Controller. A software architectural pattern for implementing user interfaces by separating concerns into three interconnected components: Model, View, and Controller.\\

\textbf{NPM:} Node Package Manager. A package manager for JavaScript, used to install libraries and manage dependencies in Node.js applications.\\

\textbf{OPA:} Open Policy Agent. A general-purpose policy engine that enables fine-grained access control across systems.\\

\textbf{Operating System (OS):} The software that manages hardware resources and provides services for computer programs.\\

\textbf{Policies:} Rules or guidelines applied in software systems to control access, security, and operations.\\

\textbf{POC:} Proof of Concept. A prototype or demonstration used to verify a concept or idea in a project.\\

\textbf{PostgreSQL:} An open-source relational database management system (RDBMS) known for its extensibility and standards compliance.\\

\textbf{Processing:} The act of executing tasks or computations on data by a computer's CPU.\\

\textbf{PR:} Pull Request, a request to bring changes in the source code of the project.

\textbf{RAM:} Random Access Memory. A type of computer memory that stores data temporarily, allowing for fast read and write operations.\\

\textbf{Token:} A digital object used to authenticate and authorize users in secure communication, often used in token-based authentication systems (e.g., JWT).\\

\textbf{TS:} TypeScript. A strongly typed programming language that builds on JavaScript, offering better tooling for large applications.\\

\textbf{Mock:}
\textbf{Beans:}
\textbf{UI:}
\textbf{DOM}
\textbf{dependencie}