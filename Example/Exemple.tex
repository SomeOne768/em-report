\documentclass[12pt,a4paper]{article}
\usepackage[utf8]{inputenc}
\usepackage[T1]{fontenc}
\usepackage[francais]{babel}
\usepackage{amsmath}
\usepackage{amsfonts}
\usepackage{amssymb}
\usepackage{graphicx}
\usepackage{textcomp}
\usepackage{listings}
\usepackage{caption}
\usepackage{subcaption}
\usepackage{lipsum}
\usepackage{enumitem}
\usepackage{algorithm}
\usepackage{algpseudocode}
\usepackage[left=3cm,right=2cm,top=2.5cm,bottom=2.5cm]{geometry}
\usepackage{hyperref}

\title{Rapport d'élève ingénieur\\
	Projet de 3ème année\\
	Filière 1 et 5 : ??? et Réseaux et sécurité informatique\\
	\textbf{Développement d'un moteur de jeu}}
\author{Présenté par: \textbf{AntoNain DEVIDAL et Abdeljalil ZOGHLAMI}}
\date{\today}


\begin{document}
	
	\begin{figure}[t]
		\centering
		\begin{tabular}{c@{\hspace{60mm}}c}
			\includegraphics[width=50mm]{Image/Logo_ISIMA_INP.png}
		\end{tabular}
	\end{figure}
	
	
	
	\maketitle
	
	
	\thispagestyle{empty}
	
	\vfill 
	\noindent Tuteur de projet: \textbf{PEREIDA Alexis}
	\hfill %\hspace{0pt}
	XX/03/2024
	\begin{flushright}
		Durée: 240 heures
	\end{flushright}
	Campus des Cézeaux. 1 rue de la Chebarde. TSA,60125. 63178 Aubière CEDEX
	
	\newpage

	\tableofcontents
	\pagebreak
	
	\section{Algorithme}
			
		\begin{algorithm}[H]
			\caption{Validation croisée}
			\label{algo:cross_validation}
			\begin{algorithmic}[1]
				\Require{$X$: ensemble de données, $K$: nombre de plis}
				\Ensure{Performance du modèle}
				\State $S \gets$ Diviser $X$ en $K$ plis de taille égale
				\State $performance\_totale \gets 0$
				\For{$i \gets 1$ to $K$}
				\State $X_{\text{train}} \gets X \setminus S[i]$
				\State $X_{\text{test}} \gets S[i]$
				\State Entraîner le modèle sur $X_{\text{train}}$
				\State Évaluer la performance du modèle sur $X_{\text{test}}$
				\State $performance\_totale \gets performance\_totale +$ performance du modèle
				\EndFor
				\State $performance\_moyenne \gets \frac{performance\_totale}{K}$
				\State \textbf{return} $performance\_moyenne$
			\end{algorithmic}
		\end{algorithm}
		
		% ou alors:		
		%\usepackage[ruled,vlined, french]{algorithm2e}
		%\begin{algorithm}[H]
		%    \SetKwInput{KwInOut}{Input-Output}%
		%    \KwIn{Un graphe de tâches avec durées.}%
		%    \KwOut{La fonction de rang $r$}%
		%    \BlankLine%
		%    \ForAll{$u\in T$}%
		%    {$\alpha(u):=|N^-(u)|$\;}%
		%    $i:=0$\;%
		%    $S_{0}:=\{debut\}$ \tcp*[h]{Le rang $0$ est réduit à $debut$}\;%
		%    \While{$S_{i}\neq \emptyset$}%
		%    {$S_{i+1}:=\emptyset$\;%
			%        \ForAll{$u\in S_{i}$}%
			%        {$r(u):=i$\;%
				%            \ForAll{$v\in N^+(u)$}%
				%            {$\alpha(v):=\alpha(v)-1$\;%
					%                \If{$\alpha(v)=0$}%
					%                {$S_{i+1}:=S_{i+1}\cup \{v\}$\;}%
					%            }%
				%        }%
			%        $i:=i+1$\;%
			%    }%
		%    \Return{$r$}%
		%    \caption{Calcul du rang des tâches.}%
		%    \label{Algo:Rang-DAG}%
		%\end{algorithm}
		
	\section{Maths}
		
		\[
		L_{\log}(1, p) = -\log(p)
		\]
		
		et cette fonction atteint une valeur proche de l'infini lorsque la probabilité $p$ se rapproche de zéro :
		
		\[
		L_{\log}(1, p) \to +\infty \quad \text{lorsque} \quad p \to 0
		\]
		
		$\sum_{0}^{n}$ $x^n$
		$\sum_{i=0}^{n}$ $x_i$
		
		
	\section{Image}
	%!h pour être sûr du positionnement
		\subsection{Forcer positionnement}
		\begin{figure}[!h]			
			\centering{\includegraphics[height=30mm]{Image/Logo_ISIMA_INP.png}}
		\end{figure}
		
		\subsection{en haut d'une page}
		Nécessite de jouer avec des 'pagebreak / newpage' parce que si pas assez de place, placement page suivante
		\begin{figure}[t]
			\includegraphics[height=30mm]{Image/Logo_ISIMA_INP.png}
			\hfill
			\includegraphics[height=30mm]{Image/Logo_ISIMA_INP.png}
			\\
			
			\centering{\includegraphics[height=30mm]{Image/Logo_ISIMA_INP.png}}
		\end{figure}
		
	\section{Ressources}
	\paragraph{Dexetify} : https://detexify.kirelabs.org/classify.html, permet de dessiner à la main le symbole que l'on recherche pour ensuite avoir son 'code' LaTeX.
	\paragraph{ChatGPT} : si une tâche ne demande pas d'intelligence comme par exemple demander à ajouter '\%' à chaque début de ligne. Une simple façon de savoir s'il sera utile: est-ce dans les cordes d'un utilisateur Mac ?

\end{document}